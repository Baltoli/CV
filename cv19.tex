%%%%%%%%%%%%%%%%%%%%%%%%%%%%%%%%%%%%%%%%%%%%%%%%%%%%%%%%%%%%%%%%%%%%%%%%%%%%%%%%
% Medium Length Graduate Curriculum Vitae
% LaTeX Template
% Version 1.2 (3/28/15)
%
% This template has been downloaded from:
% http://www.LaTeXTemplates.com
%
% Original author:
% Rensselaer Polytechnic Institute 
% (http://www.rpi.edu/dept/arc/training/latex/resumes/)
%
% Modified by:
% Daniel L Marks <xleafr@gmail.com> 3/28/2015
%
% Important note:
% This template requires the res.cls file to be in the same directory as the
% .tex file. The res.cls file provides the resume style used for structuring the
% document.
%
%%%%%%%%%%%%%%%%%%%%%%%%%%%%%%%%%%%%%%%%%%%%%%%%%%%%%%%%%%%%%%%%%%%%%%%%%%%%%%%%

%-------------------------------------------------------------------------------
%	PACKAGES AND OTHER DOCUMENT CONFIGURATIONS
%-------------------------------------------------------------------------------

%%%%%%%%%%%%%%%%%%%%%%%%%%%%%%%%%%%%%%%%%%%%%%%%%%%%%%%%%%%%%%%%%%%%%%%%%%%%%%%%
% You can have multiple style options the legal options ones are:
%
%   centered:	the name and address are centered at the top of the page 
%				(default)
%
%   line:		the name is the left with a horizontal line then the address to
%				the right
%
%   overlapped:	the section titles overlap the body text (default)
%
%   margin:		the section titles are to the left of the body text
%		
%   11pt:		use 11 point fonts instead of 10 point fonts
%
%   12pt:		use 12 point fonts instead of 10 point fonts
%
%%%%%%%%%%%%%%%%%%%%%%%%%%%%%%%%%%%%%%%%%%%%%%%%%%%%%%%%%%%%%%%%%%%%%%%%%%%%%%%%

\documentclass[margin]{res}  

% Default font is the helvetica postscript font
\usepackage{helvet}
\usepackage{url}
\usepackage{hyperref}

% Increase text height
\textheight=700pt

\begin{document}

%-------------------------------------------------------------------------------
%	NAME AND ADDRESS SECTION
%-------------------------------------------------------------------------------
\name{Bruce Collie}

% Note that addresses can be used for other contact information:
% -phone numbers
% -email addresses
% -linked-in profile

\address{\href{mailto://brucecollie82@gmail.com}{\texttt{brucecollie82@gmail.com}}\\\href{http://baltoli.github.io}{\texttt{baltoli.github.io}}\\07591461843}
\address{7/2 Ardmillan Terrace\\Edinburgh\\EH11 2JN}

% Uncomment to add a third address
%\address{Address 3 line 1\\Address 3 line 2\\Address 3 line 3}
%-------------------------------------------------------------------------------

\begin{resume}

\section{EDUCATION}
\textbf{University of Edinburgh}: CDT in Pervasive Parallelism \\
{\sl PhD} (2018--)\hfill In Progress
\\
{\sl MScR} (2017--18)\hfill
Distinction \\\\
\textbf{University of Cambridge}: Computer Science \\
{\sl MEng} (2016--17)\hfill Distinction \\
{\sl BA} (2013--16)\hfill $1^{\mathrm{st}}$ Class

\begin{format}
\title{l}\employer{r}\\
\dates{l}\location{r}\\
\body\\
\end{format}

\section{EXPERTISE}

\begin{description}
  \item[C++] Primary programing language expertise; strong proficiency and
    working knowledge of modern best practice when developing production
    applications.
  \item[Compilers] Academic and industrial experience of compiler
implementation and theory, including significant experience with LLVM.
  \item[Languages] Professional or substantial academic experience using C,
    Python, Ruby, Scala, Java, and OCaml. Able to quickly adapt to new
    languages, tools and environments.
  \item[Tools] Familiar with common software engineering tools and workflows,
    including source control (Git, SVN), project management (GitHub, JIRA) and
    CI/CD pipelines. Significant experience with remote working practices.
  \item[Research] Able to communicate effectively in person and through written
    media. High-quality, award-winning publication record across multiple top
    computer science conference venues.
\end{description}

\section{EXPERIENCE}

\employer{\textbf{Research Developer, Compilers Team}}
\location{\emph{Edinburgh, 2019--}}
\dates{}
\title{\textbf{Huawei}}
\begin{position}
Developed cutting-edge features within a production-grade compiler as part of a
larger programming languages research team. Established and maintained developer
tools integral to the team's work, and provided specialist C++ expertise.

Successfully balanced part-time employment with PhD studies in order to gain
experience working on real-world compiler technology.
\end{position}

\employer{\textbf{Core Payments Developer}}
\location{\emph{London, 2016}}
\dates{}
\title{\textbf{GoCardless}}
\begin{position}
Worked on developing critical, high-performance financial web services in Ruby, with
responsibility for developing internal libraries, performing key infrastructure
upgrades and responding to customer bug reports. Led initial public-facing work
on new open-source efforts by the core team.
\end{position}

\employer{\textbf{Presales Developer}}
\location{\emph{Cambridge, 2015}}
\dates{}
\title{\textbf{RealVNC}}
\begin{position}
Developed substantial, complex embedded software to interface a proprietary
in-car entertainment platform with Apple's CarPlay software, working with Linux
kernel modules and device drivers.
\end{position}

\clearpage

\section{PUBLICATIONS}

\par
\textbf{Modeling Black-Box Components with Probabilistic Synthesis}\\
\emph{GPCE '20, Conference Paper (Best Paper Award)}

\par
\textbf{M$^\mathbf{3}$: Semantic API Migrations}\\
\emph{ASE '20, Conference Paper}

\par
\textbf{Retrofitting Symbolic Holes to LLVM IR}\\
\emph{TyDe '20, Presentation}

\par
\textbf{Automatically Harnessing Sparse Acceleration}\\
\emph{CC '19, Conference Paper}

\par
\textbf{Type-Directed Program Synthesis and\\Constraint Generation for Library Portability}\\
\emph{PACT '19, Conference Paper}

\par
\textbf{Augmenting Type Signatures for Program Synthesis}\\
\emph{TyDe '19, Presentation}

\section{RESEARCH\\INTERESTS}

\par
\textbf{Program Synthesis}\\
Understanding and modelling the behaviour of black-box interfaces, and using
synthesized programs in novel applications of similar-program search.
Development of new algorithms and tools for synthesizing and JIT-compiling
imperative programs.

\par
\textbf{Programming Language Semantics}\\
Working at the intersection of programming languages and systems research, with
a focus on how novel domain-specific languages and tools can be applied in
practical contexts. Facilitated interdisciplinary seminar discussions to improve
the flow of ideas in this area.

\par
\textbf{Heterogeneous Compilation}\\
Compilation for parallel and heterogenous architectures, with a focus on how
legacy and future code can be automatically and transparently targeted to
emerging libraries and devices. Performance analysis and modelling of
competing library implementations for scientific code.

\par
\textbf{Verification \& Analysis}\\
Static analysis and model checking for runtime assertion systems in C programs,
aiming to improve the performance of compiler instrumentation. Development of
profiling and analysis tools for TESLA, a runtime assertion language implemented
as a compiler extension.


\end{resume}
\end{document}

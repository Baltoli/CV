% LaTeX Curriculum Vitae Template
%
% Copyright (C) 2004-2009 Jason Blevins <jrblevin@sdf.lonestar.org>
% http://jblevins.org/projects/cv-template/
%
% You may use use this document as a template to create your own CV
% and you may redistribute the source code freely. No attribution is
% required in any resulting documents. I do ask that you please leave
% this notice and the above URL in the source code if you choose to
% redistribute this file.

\documentclass[letterpaper]{article}

\usepackage{hyperref}
\usepackage{geometry}

% Comment the following lines to use the default Computer Modern font
% instead of the Palatino font provided by the mathpazo package.
% Remove the 'osf' bit if you don't like the old style figures.
\usepackage[T1]{fontenc}
\usepackage[sc,osf]{mathpazo}

% Set your name here
\def\name{Bruce Collie}

% Replace this with a link to your CV if you like, or set it empty
% (as in \def\footerlink{}) to remove the link in the footer:
\def\footerlink{}

% The following metadata will show up in the PDF properties
\hypersetup{
  colorlinks = true,
  urlcolor = black,
  pdfauthor = {\name},
  pdfkeywords = {computer science},
  pdftitle = {\name: Curriculum Vitae},
  pdfsubject = {Curriculum Vitae},
  pdfpagemode = UseNone
}

\geometry{
  body={6.5in, 8.5in},
  left=1.0in,
  top=1.25in
}


% Customize page headers
\pagestyle{myheadings}
\markright{\name}
\thispagestyle{empty}

% Custom section fonts
\usepackage{sectsty}
\sectionfont{\rmfamily\mdseries\Large}
\subsectionfont{\rmfamily\mdseries\itshape\large}

% Other possible font commands include:
% \ttfamily for teletype,
% \sffamily for sans serif,
% \bfseries for bold,
% \scshape for small caps,
% \normalsize, \large, \Large, \LARGE sizes.

% Don't indent paragraphs.
\setlength\parindent{0em}

% Make lists without bullets
%\renewenvironment{itemize}{
%  \begin{list}{}{
%    \setlength{\leftmargin}{1.5em}
%  }
%}{
%  \end{list}
%}

\begin{document}

% Place name at left
{\huge \name}

% Alternatively, print name centered and bold:
%\centerline{\huge \bf \name}

\vspace{0.25in}

\begin{minipage}{0.45\linewidth}
  104/3 Blackford Avenue \\
  Edinburgh, Scotland \\
  EH9 3ES
\end{minipage}
\begin{minipage}{0.45\linewidth}
  \begin{tabular}{ll}
    Phone: & +447591 461 843 \\
    Email: & \href{mailto:brucecollie82@gmail.com}{\tt brucecollie82@gmail.com}
  \end{tabular}
\end{minipage}

\section*{Education}

\begin{itemize}
  \item \textbf{University of Edinburgh}, Informatics (2017--)
    \begin{itemize}
      \item \textbf{PhD Pervasive Parallelism} (\emph{In Progress})
    \end{itemize}
  \item \textbf{University of Cambridge}, Computer Science (2013--2017)
        \begin{itemize}
            \item \textbf{MEng} (\emph{Distinction})
            \item \textbf{BA Hons.} (\emph{First Class})
        \end{itemize}
\end{itemize}

\section*{Research and Academic Projects}
  \begin{itemize}
    \item \textbf{Pervasive Parallelism} MScR Research Project \\
      \begin{tabular}{c|c}
        \emph{September 2017--August 2018} & \emph{University of Edinburgh}
      \end{tabular}
      \begin{itemize}
        \item In the first year of my PhD programme, I will be undertaking an
          MScR research project in the area of compiler technology and automatic
          parallelisation of programs (with particular focus on heterogeneity).
      \end{itemize}
    \item \textbf{Static Analysis with TESLA} MEng Research Project \\
      \begin{tabular}{c|c}
        \emph{November 2016--June 2017} & \emph{University of Cambridge}
      \end{tabular}
      \begin{itemize}
        \item TESLA is a framework for dynamic instrumentation of C programs
          using temporal logic assertions. As part of my MEng degree, I
          conducted research into how static analysis can be used to improve the
          performance of TESLA, enabling its use in new contexts.

        \item The project mixed theory with practice by integrating model
          checking techniques with the LLVM-based TESLA toolchain.

        \item Several promising results were obtained from the
          project---performance of existing TESLA assertions could be improved
          dramatically, and a detailed characterisation of programs suitable for
          TESLA instrumentation was carried out.

        \item The final implementation and associated report received a
          distinction.
      \end{itemize}

    \item \textbf{An Implementation of the $\pi$-Calculus} Undergraduate Final
      Project \\
        \begin{tabular}{c|c}
            \emph{October 2015--May 2016} & \emph{University of Cambridge}
        \end{tabular}
        \begin{itemize}
            \item In my final undergraduate year, my major project and
              dissertation involved the development of a programming language
              and runtime environment based on the $\pi$-Calculus process
              algebra.

            \item The project included a bytecode compiler for a language of my
              own design, as well as a full scheduling virtual machine to
              execute the bytecode. Both the compiler and virtual machine were
              written in Scala.

            \item In undertaking this project I gained strong experience in the
              theory and practice of compiler implementation, as well as an
              understanding of the problems faced when building concurrent
              systems.

            \item The final project implementation and associated dissertation
              received a first-class grade.
        \end{itemize}
  \end{itemize}
\section*{Employment}

\begin{itemize}

    \item \textbf{Core Payments Intern} GoCardless \\
        \begin{tabular}{c|c}
            \emph{June--September 2016} & \emph{London, UK}
        \end{tabular}
        \begin{itemize}
          \item While at GoCardless I worked with the company's core payments
            team to improve and develop their banking infrastructure and
            processes.
          \item Projects I was responsible for included adding new features to
            internal risk and compliance tools, upgrading legacy code to allow
            for infrastructure changes, and designing auditing methods for
            changes to payment processes.
          \item I also spent time as the team's first responder, dealing with
            problems and queries about banking processes from across the
            company.
        \end{itemize}

    \item \textbf{Mobile Presales Intern} VNC Automotive \\
        \begin{tabular}{c|c}
            \emph{June--September 2015} & \emph{Cambridge, UK}
        \end{tabular}
        \begin{itemize}
            \item At RealVNC I was primarily responsible for prototyping an
              implementation of Apple's CarPlay in-car entertainment software.

            \item My role involved developing kernel drivers to support USB
              communication with the iPhone, as well as building the user-level
              CarPlay software itself.

            \item I was responsible for undertaking preliminary research into
              how CarPlay could be integrated into the company's existing in-car
              systems, as well as the development work involved in building a
              prototype.
        \end{itemize}

    
    \item \textbf{Mobile Development Intern} University of Cambridge, Mobile Systems Group \\
		\begin{tabular}{c|c}
            \emph{June 2014--April 2015} & \emph{Cambridge, UK}
		\end{tabular}
		\begin{itemize}

            \item I built a native iOS version of an existing Android
                application (\emph{EasyM}) for collecting survey response and
                sensor data from users who take part in research studies.

            \item As well as being of active use to researchers, my work also
                served to evaluate the potential of Swift and iOS as a platform
                for mobile sensing applications related to the work of the
                research group.

            \item My primary work was carried out during an internship from
                June--September 2014, with more work being carried out part time
                from then until April 2015. At the end of my internship period I
                presented on my work to a group of academics at a research group
                meeting.

		\end{itemize}
	
    % \item \textbf{iOS Developer} PenPalWorld \\
		% \begin{tabular}{c|c}
			% \emph{July 2013--April 2015} & \emph{Remote}
		% \end{tabular}
		% \begin{itemize}

    %         \item At PenPalWorld I created an iOS app for a an existing social
    %             networking site with over a million users. Creating the app also
    %             involved builing a custom web API to interface with the site's
    %             existing database.

    %         % \item Throughout the project I worked closely with the owner
    %         %     and founder of PenPalWorld to produce a quality end product.

    %         % \item On launch my app was used by approximately 150 unique users
    %         %     per day, and my responsibilities were expanded to include
    %         %     managing user satisfaction with the product.

% \end{itemize}
	
    %\item \textbf{Assistant iOS Developer} Mostly Serious \\
%		\begin{tabular}{c|c}
%			\emph{July - August 2012} & \emph{Remote}
%		\end{tabular}
%		\begin{itemize}
%
 %           \item I worked remotely with Mostly Serious to assist their
  %              developers in the creation of a time tracking appliction for the
   %             iPhone.
%
%		\end{itemize}
	
    %\item \textbf{Bioengineering Intern} Heriot Watt University, Chemical Engineering Department \\
%		\begin{tabular}{c|c}
%			\emph{July - August 2012} & \emph{Edinburgh, UK}
%		\end{tabular}
%		\begin{itemize}
%
%            \item During my 5\textsuperscript{th} year at school I took part in
%                the Nuffield bursary scheme. I spent the summer working to
%                create an automated feedback system for controlling the growth
%                of yeast in a fermenter.
%
%		\end{itemize}
\end{itemize}

\section*{Other Projects}

\begin{itemize}


    \item \textbf{Webmaster} Trinity Hall June Event \\
        \begin{tabular}{c|c}
            \emph{October 2014--June 2015} & \emph{Cambridge, UK}
        \end{tabular}
        \begin{itemize}
            \item In my second year of university I took on the role of
              Webmaster for the Trinity Hall June Event 2015, the largest
              capacity event in Cambridge University's famous May Week (with
              2000 people attending the event).

            \item I was responsible for web development, as well as managing the
              use of the event's ticketing software. This involved both
              practical and technical skills---I ran a team of 6 people on the
              night of the event to control admission, as well as writing custom
              ticket generation and distribution software.
            
            \item The role required me to work effectively on a large team with
              a wide range of interests and responsibilities.
        \end{itemize}

\end{itemize}

% Footer
%\begin{center}
%  \begin{footnotesize}
%    Last updated: \today \\
%    \href{\footerlink}{\texttt{\footerlink}}
%  \end{footnotesize}
%\end{center}

\end{document}
